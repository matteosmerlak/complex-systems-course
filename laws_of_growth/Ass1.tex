\documentclass[english, 11pt]{article}
\usepackage{notes}
\usepackage{answers}

\Newassociation{sol}{Solution}{ans}
\newtheorem{Ex}{Exercise}


% Uncomment these for a different family of fonts
% \usepackage{cmbright}
% \renewcommand{\sfdefault}{cmss}
% \renewcommand{\familydefault}{\sfdefault}

\newcommand{\thiscoursecode}{Laws of biological growth?}
\newcommand{\thiscoursename}{Complex Systems}
\newcommand{\thisprof}{Matteo Smerlak}
\newcommand{\me}{Matteo Smerlak}
\newcommand{\thisterm}{AIMS - Ghana 2017}
%\newcommand{\website}{MYWEBSITE.COM}

% Headers
\rhead{\thisterm}
\lhead{\thiscoursename}


% Begin Document
\begin{document}

  % Notes front
\begin{center}

\textbf{\Huge{\noun{\thiscoursecode}}}{\Huge \par}\vspace{0.2in}

{\large{\noun{\thiscoursename}}}\\ \vspace{0.1in}

  {\noun \thisprof} \ $\bullet$ \ {\noun \thisterm} 
  
  \end{center}
  
\vspace{1cm}


This assignment should be returned as a commented Jupyter notebook (.ipynb file). Go to \url{http://jupyter.org} to install Jupyter or run it in your browser (\url{https://try.jupyter.org}). 

\medskip
The datasets are on my GitHub page, at \url{https://github.com/msmerlak/complex-systems-course/}

\medskip
If your response happens to be inspired by online content, say it clearly, tell us where you found the content,  and show us that you challenged yourself in the process. 

\bigskip

\begin{Ex}
The dataset growth\textunderscore scaling.csv contains information about the growth dynamics of 1500+ living species, both  prokaryotes and eukaryotes. 

\begin{enumerate}
	\item Identify a mathematical function $m(t)$ which could plausibly represent the growth curve of a living being (i.e. its mass as a function of time from birth to death/replication) and plot it, with labelled axes, plot label, etc. Explain why your choice of $m(t)$ is sensible and what approximations underlies it, if any. 
	\item  Compute within your chosen model (analytically if you can,  numerically else) the time $t^*$ when the organism grows fastest, i.e. when $dm/dt$ reaches its maximum. Add this point on your plot of $m(t)$. 
	\item In the dataset, the value $x_g$ corresponds to measured values of $m(t^*)$ (in grams) and $y_g$ to $dm/dt(t^*)$ (in grams per year) respectively. What relationship do you expect to hold between $x_g$ and $y_g$?
	\item Test this assumption with a suitable regression package (e.g. \textbf{from} scipy.optimize \textbf{import} curve\textunderscore fit).
	\item Plot the data again, this time on log-log axes. Does a pattern emerge?
	\item Explain how a power-law relationship of the form $y=ax^k$ looks on log-log axes. 
	\item Conclude: what is the relationship between growth and mass at maximum growth across the tree of life? Estimate the relevant parameters by a suitable fitting procedure (with confidence intervals, of course). 
\end{enumerate}

\end{Ex}


\begin{Ex}
		The dataset growth\textunderscore curves.csv contains empirically measured growth curves $m(t)$ for $13$ different animals.
		\begin{enumerate}
			\item How similar do you expect these growth curves to be, given the species chosen?
			\item Plot all $13$ curves (as always with labelled axes etc.). How similar are they?
			\item It has been argued that the equation $$\frac{dm}{dt}=am^{3/4}-bm$$ provides a good model of the growth dynamics of animals. Solve this differential equation. How many free parameters does the solution have?
			\item Estimate these parameters for each species using a suitable fitting procedure. 
			\item Compute the limiting value $M=\lim_{t\to\infty}m(t)$ as a function of the  parameters. Estimate $M$ for each species. 
			\item Add the following two columns to the dataset: $r=(m/M)^{1/4}$ and $\tau=(at/4M^{1/4})-\ln[1-(m_0/M)^{1/4}]$. Plot $r$ vs. $\tau$ for all species on a single plot and comment on the result.
		\end{enumerate}
\end{Ex}


\begin{Ex}
Do the results presented in this assignment qualify as ``laws of complex systems''? Share your thoughts.  	
\end{Ex}



  %%%%%%%%%%%%%%%%%%%%%%%%%%%%%%%%%%%%%%%%%%%%%%%
  \end{document}
